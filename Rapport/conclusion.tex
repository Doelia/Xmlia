\chapter{Perspectives et conclusions}

\section{Perspectives}
Les délais du projet et le fait que sa réalisation soit en complément des cours nous a poussé a faire l'impasse sur des fonctionnalités intéressantes, car notre but était avant tout d'arriver à un prototype fonctionnel, auquel des ajouts ont été faits. Certaines de ces fonctionnalités seront donc développées dans cette partie du rapport.
\paragraph{}

Une fonctionnalité qui d'ailleurs avait été évoquée dans le sujet en tant que fonctionnalité optionnelle est l'affichage du fichier XML dans un navigateur Web, avec l'ajout d'une feuille de style permettant de définir la disposition et l'affichage des données dans le navigateur. Cela aurait pu fournir un affichage similaire à celui d'HTML pour ce qui est des données et de CSS pour sa feuille de style.
\paragraph{}

Une autre fonctionnalité majeure aurait pu être apportée : la modification de plusieurs documents à la fois. Cela aurait permis de ne lancer qu'une seule fois l'application pour plusieurs fichiers et donc être plus organisé et consommer moins de ressources. Pour ce qui est de la méthode d'utilisation, deux manières auraient pu être possibles ; soit un affichage moderne avec des onglets pour chaque document ouvert qui aurait requis un changement du support des schémas, soit un affichage plus ancien avec un système de type MDI (Multiple Document Interface) où chaque document ouvert serait apparu comme une sous-fenêtre de l'application principale.
\paragraph{}

Dans une approche complètement différente, nous aurions pu imaginer une application beaucoup plus axée sur l'affichage graphique avec un système plus proche de l'utilisateur novice, avec des actions simples à réaliser depuis un panel léger, en s'éloignant du principe d'éditeur de texte de base.

\section{Conclusions}

\subsection{Le projet}
Le projet était en lui-même simple mais était très complet. Nous nous sommes rendu compte au fil du temps que le projet s'adaptait très bien au contexte de notre cursus universitaire. Il nous a permis d'augmenter très largement notre pratique de la programmation. (Programmation orientée objet, parcours d’arbre, validation syntaxique, expressions régulières)


\subsection{Les outils}
D'un point de vue technique, le choix du langage (C++) et de sa librairie (Qt) s’est avéré être un bon choix. Le projet se prêtait très bien à l’apprentissage et les modules proposés par Qt comblaient bien nos besoins.

\paragraph{}
Aucun de nous quatre ne connaissait cette librairie mais nous étions conscients de la difficulté à l'appréhender. C'est ainsi que plus gros du temps passé fut celui d'apprendre à utiliser Qt. D'un point de vue pédagogique nous sommes très satisfaits de ce choix, bien que les débuts furent fort fastidieux, aujourd'hui nous pouvons considérer que nous avons effectué notre première approche sur cette célèbre librairie.
\paragraph{}
Si nous devions réaliser à nouveau un projet similaire avec Qt, nous estimons que le temps de développement serait beaucoup plus court.

\subsection{Groupe de travail}
Ce projet était pour nous quatre le premier réalisé dans ces conditions. Nous ne nous connaissions tous pas et nous n'avions pas tous réalisé le même parcours ; nous avions tous des compétences différentes.
\paragraph{}

Le système de billet nous a facilité le travail de répartition des tâches en termes de communication. Toutes les tâches étaient découpées en petites parties que chacun pouvait réaliser en fonction de ses possibilités. Ceci permettait à tous les membres du groupe de participer même si tout le monde n’effectuait pas nécessairement des choix de développement.
\paragraph{}

Les premières réunions furent difficiles mais peu à peu nous avons pris conscience des compétences de chacun et nous avons pu réaliser les bons choix pour arriver au bout du projet dans de bonnes conditions.



