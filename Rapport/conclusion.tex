\chapter{Perspectives et conclusions}
	\section{Perspectives}
Les délais du projet et le fait que sa réalisation soit en complément des cours nous a poussé a faire l'impasse sur des fonctionnalités intéressantes, car notre but était avant tout d'arriver à un prototype fonctionnel, auquel des ajouts ont été faits. Certaines de ces fonctionnalités seront donc développées dans cette partie du rapport.
\paragraph{}

Une fonctionnalité qui d'ailleurs avait été évoquée dans le sujet en tant que fonctionnalité optionnelle est l'affichage du fichier XML dans un navigateur Web, avec l'ajout d'une feuille de style permettant de définir la disposition et l'affichage des données dans le navigateur. Cela aurait pu fournir un affichage similaire à celui d'HTML pour ce qui est des données et de CSS pour sa feuille de style.
\paragraph{}

Une autre fonctionnalité majeure aurait pu être apportée : la modification de plusieurs documents à la fois. Cela aurait permis de ne lancer qu'une seule fois l'application pour plusieurs fichiers et donc être plus organisé et consommer moins de ressources. Pour ce qui est de la méthode d'utilisation, deux manières auraient pu être possibles ; soit un affichage moderne avec des onglets pour chaque document ouvert qui aurait requis un changement du support des schémas, soit un affichage plus ancien avec un système de type MDI (Multiple Document Interface) où chaque document ouvert serait apparu comme une sous-fenêtre de l'application principale.
\paragraph{}

Dans une approche complètement différente, nous aurions pu imaginer une application beaucoup plus axée sur l'affichage graphique avec un système plus proche de l'utilisateur novice, avec des actions simples à réaliser depuis un panel léger, en s'éloignant du principe d'éditeur de texte de base.

	\section{Conclusions}