\chapter{Organisation du projet}

  \section{Organisation du travail}

    \subsection{Réunions de travail}
Une réunion hebdomadaire était tenue tous les mardis, d'abord à 13h15, rapidement décalée à 11h20. Certaines réunions exceptionnelles ont pû être rajoutées le jeudi.

    \subsection{Répartition des tâches}
Les tâches étaient assignées au dévouement ou, si certaines tâches n'étaient pas assignées et que des personnes ne s'étaient pas dévouées, par l'encadrant.

    \subsection{Planification du développement}
On peut analyser la méthode de développement utilisée de la manière suivante :
\begin{itemize}
	\item analyse globale du problème~;
	\item analyse des structures de données nécessaires à la résolution du problème~;
	\item implémentation des structures et de leurs fonctions associées~;
	\item analyse en profondeur du problème et écriture des algorithmes nécessaires à sa résolution~;
	\item implémentation des algorithmes en s'appuyant sur les structures déjà implémentées~;
	\item construction de l'application finale à partir des briques logicielles implémentées.
\end{itemize}

    \subsection{Élection d’un chef de projet}
Aucun chef de projet n'a été élu, les décisions ont été prises lors des réunions, où tout le monde pouvait faire des propositions qui étaient alors débattues.
Le groupe de travail était suffisamment restreint pour qu'un tel système fonctionne.

    \subsection{Gestion du groupe d’étudiants}
La gestion du groupe s'est faite par courriel, lors des réunions et lors de rencontres hors du cadre du projet.
Elle s'est avérée compliquée et de nombreux problèmes de communication et tout particulièrement de non comunication et d'absentéisme sont apparus et ont retardé le projet.

  \section{Choix des outils de développement}

    \subsection{Choix du langage}
Le langage initialement proposé par notre encadrant était le C. Ce langage étant le mieux connu de toute l'équipe, il avait alors un avantage subjectif mais réel.

Il s'avère que le C répondait en outre assez bien aux besoins du projet, qui n'a nécessité que des structures de données assez simples.

La question posée a été celle de la version du langage à utiliser : C89 ou C99 ?
Au final le C89 a été choisi, afin de forcer l'équipe à utiliser un langage plus strict.

    \subsection{Choix de l'éditeur}
Le C étant un langage permettant un choix extrêmement souple d'éditeurs, il a été décidé de ne pas associer le projet à un éditeur particulier et de favoriser l'utilisation d'outils simples.
Ainsi le choix d'un simple éditeur de texte et d'un émulateur de terminal a été fait par la majorité, sinon la totalité, des membres de l'équipe.

    \subsection{Choix du compilateur}
GCC est le compilateur C utilisé par tous à la faculté, de plus certains membres de l'équipe sont attachés aux outils GNU, ainsi son choix a été fait sans se poser de questions particulières.

Le choix a été fait sur le tard d'en activer les options \verb|--ansi| et \verb|--pedantic| pour plus de rigueur dans le code.

    \subsection{Choix du débogueur}
Pour des raisons similaires à celles du choix du compilateur, GDB a été choisi pour nos besoins de déboguage.

    \subsection{Choix du gestionnaire de projet et du gestionnaire de versions}
Le service informatique de la faculté a mis à disposition des étudiants et professeurs un service d'hébergement et de gestion de projet utilisant GitLab aux alentours du démarrage du projet.
Il a rapidement été décidé de l'adopter, ainsi Git a été choisi comme gestionnaire de version car c'est celui utilisé par GitLab.

    \subsection{Choix des outils d'analyse}
Aucun outil d'analyse n'a été utilisé.

    \subsection{Choix des outils de documentation}
Concernant la documentation du code, Doxygen a été proposé par notre encadrant et a été accepté par les membres de l'équipe.
Quant au rapport, le choix de \LaTeX a été fait car il permet de modulariser et de simplifier le travail à plusieurs sur un document par rapport à des outils de traitement de texte plus classiques utilisant un format de fichier binaire et monolithique.
