\chapter{Analyse du projet}
	\section{Contexte}
	Le langage Extensible Markup Language ou XML est utilisé à des fins de stockage de données, et est structuré par un schéma qui lui est associé, il permet de definir la structure et le type de contenu du document, en plus de permettre de verifier la validité du document.

	Généralement, les fichiers XML sont générés par un programme quelconque dans le but d'échanger des données ou de les stocker, XML faisant office de plateforme commune. Mais on peut également utiliser un éditeur de texte basique pour créer de toute pièce un document XML, avec des fonctionnalités propre à un éditeur de texte, sans fonctionnalités spécialement prévues pour XML.
	
	C'est dans ce contexte que des solutions logicielles d'éditeur XML ont vu le jour : un éditeur de texte qui possède des fonctionnalités permettant une écriture d'un fichier XML beaucoup plus rapide et efficace, le tout avec un contrôle des erreurs. Plusieurs solutions sont déjà proposées, certaines étant payantes (OxygenXML) et d'autres sont gratuites et open source (Xerlin). L'objectif est ici de fournir une solution similaire aux autres logiciels.
	
	\section{Analyse des besoins fonctionnels}
	L'objectif du projet est de développer un éditeur XML multi-vues avec différentes fonctionnalités.
	
	Les fonctionnalités liées à un éditeur de texte simple devront être présentes : la possibilité de saisir manuellement au clavier l'intégralité du fichier, la création et la sauvegarde du fichier à manipuler ainsi que l'ouverture d'un fichier déjà existant dans le but de le modifier.
	
	Des fonctionnalités d'éditeur de texte avancées seront aussi présentes : coloration syntaxique et indentation automatique du code permettant ainsi une lisibilité claire des fichiers manipulés et une autocomplétion du code écrit permettant un gain de temps au cours de la frappe.

	Pour finir, l'éditeur proposera des fonctionnalités spécifiques au langage XML : validation syntaxique du fichier, vue arborescente du fichier XML avec possibilité de modification des données via cette vue et ajout d'un schéma sur lequel la validation se basera.
	
	TODO : FAIRE UN SCHEMA D'INTERFACE
	
	\section{Analyse des besoins non fonctionnels}
		\subsection{Spécifications techniques}
		Le programme devra permettre de créer des fichiers XML structurés avec un respect des normes de balisage et, s'il est défini, du schéma de données. De plus, les données saisies ou modifiées à l'aide de l'éditeur doivent rester exploitables, sans corruption du fichier original. Pour terminer, l'éditeur aura à répondre dans des durées acceptables et de manière stable, dans la mesure où la taille et la complexité des données restent raisonnables.
		
		\subsection{Contraintes ergonomiques}
                Le logiciel devra être suffisamment simple pour qu'un utilisateur connaissant déjà le fonctionnement du XML puisse l'utiliser sans être bloqué par une courbe d'apprentissage trop élevée. On utilisera pour cela des icônes claires et des textes explicatifs.
                Un utilisateur avancé pourra augmenter sa productivité en utilisant les raccourcis clavier disponibles et pourra gagner de temps en réduisant les transitions souris/clavier.
