\chapter{Introduction}

Lorsqu'un programme nécessitait un stockage des données complexes et ordonnées, son créateur décidait de la manière dont les données étaient organisées en mémoire, permettant donc de définir le comportement spécifique de son application au moment de la lecture de données enregistrées. C'est dans ce contexte que le problème de la communication de données entre deux applications ou plus s'est posé. En effet, chaque programme avait sa manière d'interpréter des données stockées et les organisait de manière spécifique, la communication directe n'était donc pas possible, il fallait donc trouver un moyen intermédiaire afin de convertir les données destinées à une application vers un format lisible par un autre programme. Sauf que créer cet intermédiaire engendrait d'important coûts en termes de développement, d'autant plus que si une nouvelle application avait besoin de ces données, il aurait à nouveau fallu recréer un intermédiaire spécifique.
\paragraph{}

C'est ainsi que le principe d'une structure de données commune a émergé et que les langages de balisage se sont popularisés, permettant en plus d'avoir une structure stricte et normalisée. XML ou "Extensible Markup Language" fait partie de ces langages. Ce langage a de plus pour caractéristique d'être, d'où son nom, extensible, c'est-à-dire que les désignations des balises ne sont pas fixes, elles sont définies spécifiquement pour les données sauvegardées. Pour finir, il est possible de définir un XML Schema afin de restreindre et de contrôler la structure même du document XML, afin de vérifier s'il est écrit de la bonne manière et valide en termes d'organisation.
\paragraph{}

Le projet qui nous a été confié consiste à concevoir et à développer une application faisant office d'éditeur XML permettant donc à n'importe quel utilisateur qui utilisera l'interface de créer ou de modifier un document XML à sa guise, le tout en conservant le respect des normes dictées par le standard XML, incluant donc tout le processus de validation des données.
\paragraph{}

Après avoir exposé l'analyse menée pour étudier le projet même, ses spécifications et ses besoins, nous présenterons le rapport d'activité rendant compte des méthodes de travail que nous avons utilisées pour mener à bien ce projet. Le rapport technique décrit les choix de conception ainsi que des extraits plus pratiques, expliquant des portions de code. Et enfin, après avoir présenté le manuel d'utilisation de l'application, nous conclurons en exposant des perspectives d'amélioration du logiciel.