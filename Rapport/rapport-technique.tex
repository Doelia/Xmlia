\chapter{Rapport technique}

\section{Conception}


\section{Architecture de l'application}
\subsection{Le modèle}
\subsection{L'arborescence}

\subsection{L'éditeur de texte}
L'éditeur de texte est décomposé en deux parties, l'éditeur de schéma et d'éditeur XML. Ce sont en réalité des spécialisation de la classe TextEditor.
\subsubsection{La classe TextEditor}
C'est ici que sont toutes les méthodes servant à l'édition du texte en général, telle que l'insertion de texte, l'indentation ou la coloration.
Nous avons fait le choix de représenter les données de l'éditeur de texte simplement par une QString, pas de référence vers la position d'un noeud ou d'information supplémentaire comme sa taille ou la délimitation de son contenu.
Un noeud étant identifé par son chemin depuis la racine, il faut reconstruire l'arborescence XML à partir de la QString. Cela se fait à travers la classe QXmlStreamReader qui s'utilise de la manière suivante :
\lstset{language=C++, inputencoding=utf8}
\begin{lstlisting}
  QXmlStreamReader xml(text->toPlainText());
  while(!xml.atEnd())
  {
    if(xml.isStartElement())
    {
      //traitement
    }
    else if(xml.isEndElement())
    {
      //traitement
    }
  }
\end{lstlisting}
On utilise une structure de pile lors de parcours de l'arborescence. On empile l'indice du sommet lorsque l'on rencontre une balise ouvrante et on dépile lorsque l'on rencontre une balise fermante.
On parvient ainsi à se déplacer et à se repérer dans la QString.

Voici le pseudo code de l'algorithme permettant de parcourir l'arbre pour se rendre sur le noeud désiré.

\begin{verbatim}
//noeud que l'on doit trouver dans la QString
var noeudCible
var pile

//Parseur xml de Qt
var xml

Tant que l'on a pas parcouru tout l'arbre
    Si on rencontre une balise ouvrante
        Si la dernière balise rencontrée est une balise ouvrante
            Empiler 0 dans pile
        Sinon
            //c'est que l'on a atteint le fils suivant
            Incrementer le sommet de la pile de 1
        Fin Si
    Sinon Si on rencontre une balise fermante
        Dépiler pile
    Fin Si
    
    Si noeudCible = pile
        Appeler la fonction qui traitera le noeud
        //on arrête le parcours de l'arbre
        retourner
    Fin Si
   
    Sauvegarder la derière balise renconcrée
    Aller à la balise suivante
Fin Tant Que
\end{verbatim}

Cette méthode de parcours de l'arbre est appelée lorsqu'un noeud est inséré, déplacé ou supprimé dans l'arborescence. Elle est utlisée de la manière suivante :


\subsection{Le logger}

\section{Résultat}
