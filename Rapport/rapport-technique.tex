\chapter{Rapport technique}

\section{Conception}


\section{Architecture de l'application}
\subsection{Le modèle}
\subsection{L'arborescence}

\subsection{L'éditeur de texte}

L'éditeur de texte est décomposé en deux parties, l'éditeur de schéma et d'éditeur XML. Ce sont en réalité des spécialisation de la classe TextEditor, mais parlons d'abord de l'éditeur de texte en général.
\subsubsection{La classe TextEditor}
C'est ici que sont toutes les méthodes servant à l'édition du texte en général, telle que l'insertion de texte, l'indentation ou la coloration. L'éditeur de texte est une sous classe de QTextEdit, qui fournit toutes les fonctionnalités basiques d'une éditeur de texte riche, telles que la sélection de texte, la copie, l'annulation etc. On peut intégrer plusieurs fonctionnalités à un QTextEdit comme par exemple la coloration syntaxique. Nous parlerons dans un premier temps des fonctionnalités propres à notre éditeur de texte, puis nous verrons plus en détail chacune de ses implémentations, à savoir l'éditeur XML et l'éditeur de schéma.

\paragraph{}
Qt propose un QCompleter, il ne peut cependant pas s'intégrer et proposer l'auto-completion dans un QTextEdit. Même s'il ne s'intègre pas dans un QTextEdit, le QCompleter peut tout de même proposer une suggestion si on lui fourni un début de mot. On peut donc récupérer le mot sous le curseur, le donner au QCompleter et insérer le mot complété dans notre éditeur de texte. Cela se fait de la manière suivante :

\begin{lstlisting}
  //on donne le mot sous le curseur au QCompleter
  completer->setCompletionPrefix(textUnderCursor());
  //on recupere le mot propose
  QString completion = completer->currentCompletion();
  QString currentWord = textUnderCursor();

  //si un mot est sous le curseur et qu'il n'est pas deja complete
  if(currentWord.size() > 0 && currentWord.size() < completion.size())
  {
    QTextCursor cursor = text->textCursor();
    //sauvegarde de la position actuelle du curseur
    int pos = cursor.position();
    //deplacement jusqu'a la fin du mot
    while(word.indexIn(cursor.selectedText()) == 0)
    {
      cursor.movePosition(QTextCursor::Left, QTextCursor::KeepAnchor);
    }
    //insertion de la partie qui n'est pas deja ecrite
    cursor.insertText(completion.right(completion.size() - currentWord.size()));
    //le curseur est replace a sa posision originelle
    cursor.setPosition(pos, QTextCursor::KeepAnchor);
    text->setTextCursor(cursor);
\end{lstlisting}

Il n'est pas possible d'intégrer directement l'auto-completion, mais on peut en revanche facilement lier un colorateur syntaxique à un QTextEdit. Qt propose une classe abstraite QSyntaxHighLighter qu'il faut implémenter en y indiquant nos règles de coloration. Le QTextEdit appelle automatiquement la méthode \textit{highlightBlock} du QSyntaxHighLighter dans laquelle on aura indiqué nos règles.

\begin{lstlisting}
  void TextHighLighter::highlightBlock(const QString &text)
  {
    setCurrentBlockState(previousBlockState());
    //l'ordre est important
    //il ne faut pas appliquer de coloration a l'interieur
    //il faut donc colorer les commentaires en premier
    for (int i = 0; i < text.length(); ++i)
    {
      //colore les commentaires
      if(cComment(last, text, i));
      //colore le texte entre quotes
      else if(cQuote(last, text, i));
      //colore les attributs
      else if(cInMarkupAttr(last, text, i));
      //colore les balises
      else if(cMarkup(last, text, i));
    }
  }
\end{lstlisting}

Voici par exemple la méthode qui permet de colorer les commentaires :
\begin{lstlisting}
  bool TextHighLighter::cComment(int &last, const QString &text, int i)
  {
    //si on se trouve deja dans un commentaire
    if(currentBlockState() == COMMENT_STATE)
    {
      //si on trouve la fin du commentaire
      if (text.mid(i, 3) == "-->")
      {
        //on colore entre last et la fin du commentaire
        setTextColor(last, i + 4, Qt::gray);
        setCurrentBlockState(DEFAULT_STATE);
      }
      setTextColor(last, i + 1, Qt::gray);
      return true;
    }
    //si on trouve le debut d'un commentaire
    else if (text.mid(i, 4) == "<!--")
    {
      //on sauvegarde la position du debut de commentaire
      last = i;
      //on marque que l'on est dans un commentaire
      setCurrentBlockState(COMMENT_STATE);
      return true;
    }
    return false;
  }
\end{lstlisting}

\subsubsection{La classe XmlEditor}

C'est donc une sous classe de TextEditor. Nous avons fait le choix de représenter les données de l'éditeur de texte simplement par une QString, pas de référence vers la position d'un noeud ou d'information supplémentaire comme sa taille ou la délimitation de son contenu.
Un noeud étant identifié par son chemin depuis la racine, il faut reconstruire l'arborescence XML à partir de la QString. Cela se fait à travers la classe QXmlStreamReader qui s'utilise de la manière suivante :
\begin{lstlisting}
  //creation d'un QXmlStreamReader affecte du texte du QTextEdit
  QXmlStreamReader xml(text->toPlainText());
  while(!xml.atEnd())
  {
    //detection d'une balise ouvrante
    if(xml.isStartElement())
    {
      //traitement
    }
    //detection d'une balise fermante
    else if(xml.isEndElement())
    {
      //traitement
    }
  }
\end{lstlisting}

On utilise une structure de pile lors de parcours de l'arborescence. On empile l'indice du sommet lorsque l'on rencontre une balise ouvrante et on dépile lorsque l'on rencontre une balise fermante.
On parvient ainsi à se déplacer et à se repérer dans la QString.

Voici le pseudo code de l'algorithme permettant de parcourir l'arbre pour se rendre sur le noeud désiré.

\begin{verbatim}
//noeud que l'on doit trouver dans la QString
var noeudCible
var pile

//Parseur xml de Qt
var xml

Tant que l'on a pas parcouru tout l'arbre
    Si on rencontre une balise ouvrante
        Si la dernière balise rencontrée est une balise ouvrante
            Empiler 0 dans pile
        Sinon
            //c'est que l'on a atteint le fils suivant
            Incrementer le sommet de la pile de 1
        Fin Si
    Sinon Si on rencontre une balise fermante
        Dépiler pile
    Fin Si
    
    Si noeudCible = pile
        Appeler la fonction qui traitera le noeud
        //on arrête le parcours de l'arbre
        retourner
    Fin Si
   
    Sauvegarder la derière balise renconcrée
    Aller à la balise suivante
Fin Tant Que
\end{verbatim}

Cette méthode de parcours de l'arbre est appelée lorsqu'un noeud est inséré, déplacé ou supprimé dans l'arborescence. Elle est utlisée de la manière suivante :

\begin{lstlisting}
  //methode appelee lorsque l'utilisateur renomme un noeud dans l'arborescence
  //on passe en parametre un pointeur de fonction
  xmlEditor->parseDom(n, n.nodeName(), QString(newName), &XmlEditor::updateNodeName);
\end{lstlisting}

Voici l'implémentation de l'appel de fonction dans le pseudo code du parcours de document :
\begin{lstlisting}
  
  //si le chemin du node courant est le meme que celui
  //du sommet passe en parametreif(cmpVectors(path, nodePath))
  {
    //on appelle la fonction passee en parametre
    //elle traitera le node avec des effets de bord
    if((this->*function)(nbFound, begin, end, c, oldName, newName, xml))
    {
      //on arrete le parcours de l'arbre si la fonction retourne true
      return;
    }
  }
\end{lstlisting}

Voici le prototype de la fonction \textit{parseDom} avec la fonction qu'elle prend en paramètre et les fonctions de manipulation de noeud qu'elle peut prendre en paramètre.

\begin{lstlisting}
  //parse le dom jusqu'a trouver le node target
  void parseDom(QDomNode &target, QString oldName, QString newName, bool (XmlEditor::*function)
  (int &nbFound, int &begin, int &end, QTextCursor &c, QString oldname, QString newname, QXmlStreamReader &xml));
  
  //remplace le nom de la balise oldName par newName
  bool updateNodeName(int &nbFound, int &begin, int &end, QTextCursor &c, QString oldName, QString newName, QXmlStreamReader &xml);
  //supprime le noeud
  bool deleteNode(int &nbFound, int &begin, int &end, QTextCursor &c, QString oldName, QString newName, QXmlStreamReader &xml);
  //sauvegarde les donnees du noeud dans le cas d'un deplacement
  bool saveNodeData(int &nbFound, int &begin, int &end, QTextCursor &c, QString oldName, QString newName, QXmlStreamReader &xml);
  //insere un noeud
  bool insertNodeText(int &nbFound, int &begin, int &end, QTextCursor &c, QString oldName, QString newName, QXmlStreamReader &xml);
\end{lstlisting}
Ces fonctions manipulent le texte à travers un QTextCursor qui permet de naviguer dans un QTextEdit. Un QTextCursor permet notamment de se déplacer de mot en mot, d'aller à la fin de la ligne ou d'aller à la ligne suivante. Un outil donc indispensable dans la manipulation du texte.

\paragraph{}
La liaison vers un schéma se fait à l'aide d'expressions régulières. Le XML étant standardisé, les données auront toujours la même forme, une expression régulière semble donc le plus simple pour trouver une information dans du texte.

\begin{lstlisting}
  void XmlEditor::removeSchema()
  {
    QString s(text->toPlainText());
    //regex du lien vers un schema
    QRegExp r("\n*xmlns:xsi.*=\".*\.xsd\"\n*");
    suppression de la regex dans le texte
    s.remove(r);
    text->setText(s);
  }
\end{lstlisting}

Il est intéressant de noter que l'on utilise une fois de plus la classe QXmlStreamReader pour se déplacer dans l'arborescence XML. Le lien vers le schéma se trouve dans la balise racine, le plus simple pour la trouver est donc de parcourir le document jusqu'à la trouver.
\begin{lstlisting}
  while(!xml.atEnd())
  {
    //lorsque l'on trouve la premiere balise ouvrante
    if(xml.isStartElement())
    {
      //on deplace le curseur jusqu'a la position de la fin de cette balise
      moveCursorToLineAndColumn(c, xml.lineNumber() - 1, xml.columnNumber() - 1, false);
      //construction du lien vers le schema
      QString link("\nxmlns:xsi=\"http://www.w3.org/2001/XMLSchema-instance\"\nxsi:noNamespaceSchemaLocation=\"");
      link.append(XmlFileManager::getFileManager()->getSchemaName()).append("\"");
      //insertion du lien dans le document
      c.insertText(link);
      text->setTextCursor(c);
      return;
    }
    xml.readNext();
  }
\end{lstlisting}

En plus de créer et supprimer un lien vers un schéma, il pourrait être intéressant de l'extraire, pour la validation par exemple. Cela se fait une fois de plus grâce à une expression régulière. Il est intéressant de noter dans l'exemple suivant que nous ne gérons que les liens locaux vers un schéma et pas les lien vers un schéma distant.
\begin{lstlisting}
  QString s(text->toPlainText());
  //split du document pour se placer directement a l'endroit du lien
  QStringList l(s.split("xsi:noNamespaceSchemaLocation=\""));
  //si un lien est present dans le document
  if(l.size() > 1)
  {
    //split selon le caractere " pour ne garder que le lien
    s = l.at(1).split("\"").at(0);
    //on retourne le chemin courant du fichier XML auquel on concatene le lien que l'on vient de trouver
    return XmlFileManager::getFileManager()->getCurrentFilePath().append("/").append(s);
  }
  //on cherche s'il n'y a pas de lien distant
  l = s.split("xsi:SchemaLocation=\"");
  if(l.size() > 1)
  {
    //on notifie qu'on ne les gere pas encore s'il existe
    emit log("Cannot process http requests yet, XML will not be checked", QColor("orange"));
  }
  return "";
\end{lstlisting}
\subsubsection{La classe SchemaEditor}
La classe SchemaEditor ne rajoute pas vraiment de fonctionnalités à la classe TextEditor mise à part la génération de schéma à partir d'un document XML. Elle fonctionne de manière très simple, les balises sont extraites du document XML et sont ajoutées au schéma.

\subsubsection{La validation}
On peut utiliser la classe QXmlStreamReader pour la validation syntaxique, elle permet en effet de détecter une erreur de syntaxe et de nous indiquer la ligne.
\begin{lstlisting}
  while(!xml.atEnd())
  {
    xml.readNext();
    //si le parseur XML rencontre une erreur
    if(xml.hasError())
    {
      //affichage de la ligne de l'erreur avec son type
      emit log("Erreur ligne " + QString::number(xml.lineNumber()) + " : " + xml.errorString(), QColor("red"));
    }
  }
\end{lstlisting}


\subsection{Le logger}

\section{Résultat}
